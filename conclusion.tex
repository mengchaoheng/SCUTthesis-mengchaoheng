
\chapter{结\quad 论}
%\sethead[][{\headfont{}\thesissubject}][] % 重设页眉偶数页
%{}{{\headfont{}结\quad 论}}{} % 重设页眉奇数页
%学位论文的结论单独作为一章排写,但不加章号。
%结论是对整个论文主要成果的总结。在结论中应明确指出本研究内容的创造性成果或创新性理论(含新见解、新观点),对其应用前景和社会、经济价值等加以预测和评价,并指出今后进一步在本研究方向进行研究工作的展望与设想。
%如果不能导出应有的结论,也可以没有结论而进行必要的讨论。
本论文研究了基于ADRC的一类涵道风扇式无人机的控制分配问题,对于不同构型的涵道无人机采用了不同的分配策略。

对于单涵道,由仿真及试验结果可得出如下结论:
\begin{enumerate}
	\item 相同输入下实际飞行试验的输出和仿真模型的输出基本一致,表明所建立的涵道风扇式无人机的数学模型是较准确的。	
	\item 所设计的ADRC控制器稳定,且对系统输出的微分、总和扰动估计较准确。相比于PID控制,扰动抑制能力有所提高。
	\item 所提出的优先级分配法和直接分配法一样,品质因数为$ 100\% $,而常规的伪逆法仅为$ 71.99\% $。因此优先级法可对可达集内的任意力矩返回一个保方向的容许控制。
	\item 优先级分配法可以保证控制律中高优先级分量尽可能无误差分配,而仅在低优先级分量上产生分配误差,具体表现为使系统可一定程度上防止因操纵面约束引起的系统输出耦合,尽可能保持输出解耦,这是直接分配法和伪逆法所没有的特性。
	\item 在系统的可达力矩集较小时,将ADRC控制律按照优先级分解,然后再按照优先级进行分配,某种意义上提高了是提高了系统稳定性。	
\end{enumerate}
对于双涵道,有如下结论:
\begin{enumerate}
	\item 动态控制分配方法可根据执行器带宽分配伪控制输入。	
	\item 执行器指令的补偿降低了执行器动态的不良影响。
\end{enumerate}

本文的主要创新点是利用优先级分配方法求解单涵道控制分配问题、讨论了现有文献较少涉及的双涵道控制分配问题。事实上,优先级分配方法还可以和其他控制算法结合,只有该算法明显包含不同的组成部分或者其分量具有不同的物理意义,都可以进行分解然后再按优先级分配。例如非线性动态逆控制,用于反馈线性化的分量可以认为是高优先级分量。解决涵道风扇式无人机的底层控制问题之后,后期进行飞行模式转换、路径跟踪等顶层的规划问题的求解将会变得轻松。

在DFUAV中基于自抗扰控制应用控制分配算法,进一步改善了其姿态控制问题。但目前的工作仍然存在以下问题:
\begin{enumerate}
	\item 	线性化的操纵面模型不准确。
	\item	对执行器动态欠考虑或对执行器模型的建模不准确。
\end{enumerate}

然而,即使存在上述问题,系统依然可以正工作。究其原因,对于单涵道来说,四个操纵面有相同的气动特性,他们的气动力系数误差引起的分配误差只需要条件控制器增益就可弥补。对于双涵道来说,只要标称带宽大于实际带宽,控制分配就可以正常进行。

解决上述问题是对本文所作工作的延续。未来的工作可从如下几方面进行:

\begin{enumerate}
	\item 	针对非线性操纵面模型,使用非线性优化方法求解控制分配问题。
	\item	由本文对控制分配问题的描述很容易引出模型预测控制分配(MPCA),MPCA方法将分配问题当作一个受限系统的最优跟踪问题求解,是处理执行器动态的有效方法。在单片机运算性能日益提高的今天,有望在低成本无人机的飞控系统中普及。
\end{enumerate}